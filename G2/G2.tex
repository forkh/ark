\documentclass[12pt,a4paper,danish]{article}

\usepackage[danish]{babel}
\usepackage[utf8]{inputenc}
\usepackage[T1]{fontenc}
\usepackage{graphicx}
\usepackage{listliketab}

\begin{document}
\title{Gruppeøvelse 2}
\author{Anders Kiel Hovgaard\\Rúni Klein Hansen}
\date{20 oktober 2013}
\maketitle

\section{Introduktion}
En pipelined MIPS implementation, baseret på figur 4.60 s. 375 COD.\\
Mikroprocessoren er en 5-fase implementation, de forskellige faser er:
\textsf{InstructionFetch (IF), InstructionDecode (ID), Execution (EX), Memory (MEM),
Writeback (WB)}.\\ 
Disse instruktioner virker:
\begin{table}[h!]
  \centering
  \begin{tabular}{c|c}
    Virker& Virker ikke \\\hline
    addiu & slti        \\
    addu  & jal         \\
    slt   & jr          \\
    subu  &             \\
    and   &             \\
    andi  &             \\
    or    &             \\
    ori   &             \\
    lw    &             \\
    sw    &             \\
    beq   &             \\
  \end{tabular}
  \caption{Tabel over instruktioner der virker eller ej.}
  \label{tab:instr}
\end{table}
\\
Vi kan få \textsf{slti} at virke, med den konsekvens at \textsf{ori} ikke
virker da vi med nuværende implementation, ikke kan differentiere imellem ALUOp
kontrolsignaler. 

\end{document}
